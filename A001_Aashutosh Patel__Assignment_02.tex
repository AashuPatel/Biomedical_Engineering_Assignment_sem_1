\documentclass[a4paper,12pt]{article}
\usepackage{graphicx}
\graphicspath{{images/}}
\usepackage[x11names]{xcolor}
%\pagecolor{Cornsilk3}
%inline
{\Huge \title{Basic Biomedical Assignment-II \\Evolution of Modern Health Care System}}
\author{Aashutosh Patel \\ Roll no- 21111001}
\usepackage[landscape]{geometry}
\begin{document}
\begin{figure}
	\centering
	\includegraphics[scale=0.6]{nitlogo.png}
\end{figure}
\maketitle
\clearpage
\large
\section{Evolution of Modern Health Care System:} 	      \begin{figure}[h]       
        	\includegraphics[scale=.5 ]{img.jpeg}
\end{figure} 


Traditional healthcare systems are often criticized for their reactionary approach to medicine; health issues are addressed only once they’ve developed enough to become problematic.But the development and increased accessibility of medical technology give patients the chance to treat diseases at their onset, giving them a higher chance of successful recovery. 
\\
Like, some healthcare innovators have begun offering full-body MRI scans as part of their executive health exams. These safe and painless procedures can detect multiple abnormalities before the onset of symptoms occurs, including brain tumors, spinal deterioration, pulmonary lesions, heart disease and more. With these technologies, patients can be more confident that diseases aren’t growing in their bodies undetected.
\\
There are many such technologies, few of them are:
\clearpage
\subsection{Mobile Health:}
Mobile health is abbreviated as mHealth.
\\
The term is most
commonly used in reference to using mobile communication devices, such
as mobile phones, tablet computers and personal digital assistants (PDAs),
and wearable devices such as smart watches, for health services, information, and data collection.This field has emerged as a sub-segment
of eHealth, the use of information and communication technology (ICT),
such as computers, mobile phones, communications satellite, patient monitors, etc., for health services and information.
\\
mHealth can be used for:

1. Collecting community and clinical health data.

2. Real-time monitoring of patient vital signs.

3. Providing training to health workers.

\\

\subsection{Improvig Artificial Pacemaker:}
The artificial pacemaker,it is used for a century ,more than million peope use them and still it is a very critical and important component of modern health system. By delivering electrical
impulses to heart muscle chambers, they can prevent or correct life-threatening
heart arrhythmias and one of its core functionality is to be monitorable from distance(remote monitoring). traditionally this function wasn't simple, there were complex interface that could only be handeled by professionals but by enabling pacemakers with Bluetooth technology, they can be linked with smartphone-based mobile apps that patients better understand and utilize.
\\
This way modern tech is helping to complement the technologies which we are using from century.
\clearpage

\subsection{Cloud Computing in Health Sector:}
In healthcare, cloud computing has proved to be useful for both providers and patients. This tech innovation is well known for lowering operational costs for medical institutions and ensuring top-notch care for patients.The adoption of cloud computing is in its early stages in the medical sector. But such adoption opens up many opportunities for healthcare future.
\\

Benefits of Cloud Computing in Health Sector :-

1.Cost-effectiveness:

The on-demand infrastructure and resources provided by cloud computing eliminate the need 
\\ 
  for expensive in-house hardware.
\\

2.Advanced analytics:

Current technologies allow for processing massive amounts of data within minutes. Applying artificial intelligence and machine learning algorithms to analyse cloud records can facilitate medical research, predict pandemic outbreaks and find the causes of diseases. Computer algorithms can also analyse the components that are most effective for a specific condition or a particular patient. Advanced analytics can significantly contribute to personalised care plans and faster treatment.
\\

3.Active collaboration:

Healthcare data comes from various sources, such as medical records and imaging, and it is traditionally stored on paper. Cloud storage allows users to collect information in one place and access it when needed.



\end{document}
\documentclass[a4paper,12pt]{article}
\usepackage{graphicx}
\graphicspath{{images/}}
\usepackage[x11names]{xcolor}
%\pagecolor{Cornsilk3}
%inline
{\Huge \title{Basic Biomedical Assignment-I \\ Medical Devices}}
\author{Aashutosh Patel \\ Roll no- 21111001}
\usepackage[landscape]{geometry}
\begin{document}
	\begin{figure}
		\centering
		\includegraphics[scale=0.6]{nitlogo.png}
	\end{figure}
	\maketitle
	\clearpage
	\large
	\section{Teeth Whitening System:} 	\begin{figure}[h]
		\includegraphics[scale=.6 ]{Teeth.png}
	\end{figure} 
Tooth whitening or tooth bleaching is the process of lightening the colour of human teeth. Whitening is often desirable when teeth become yellowed over time for a number of reasons, and can be achieved by changing the intrinsic or extrinsic colour of the tooth enamel. The chemical degradation of the chromogens within or on the tooth is termed as bleaching.


Hydrogen peroxide is the active ingredient most commonly used in whitening products and is delivered as either hydrogen peroxide or carbamide peroxide. Hydrogen peroxide is analogous to carbamide peroxide as it is released when the stable complex is in contact with water. When it diffuses into the tooth, hydrogen peroxide acts as an oxidising agent that breaks down to produce unstable free radicals. In the spaces between the inorganic salts in tooth enamel, these unstable free radicals attach to organic pigment molecules resulting in small, less heavily pigmented components. Reflecting less light, these smaller molecules create a "whitening effect"

For a good whitening treatment, dentists should correctly diagnose because time exposure and the concentration of the bleaching compound, determines the tooth whitening endpoint.  
\\
\\

Types of Teeth Whitening System:
\\

1.Laser Teeth Whitening:
\\
Laser teeth whitening treatment is a cosmetic laser dentistry procedure that adds the use of a laser to in-office teeth whitening.
The procedure is completed in a dental office. It involves placing a concentrated whitening gel on your teeth and then using a laser to heat it up, which whitens your teeth quickly.
\\
There are few simple steps for this process:

-A plastic or rubber guard is placed in mouth to keep it open.

-A protective layer is formed on gums to protect it from bleaching.

-Whitening gel  is applied.

-Laser pen is used to activate this gel.

-After few minute the gel is removed with a small vacuum .

-Mouth is rinsed and protective layer is removed.
\\
    
    2.Strips and gels:
    \\
    The plastic whitening strips contain a thin layer of peroxide gel and are shaped to fit the buccal surfaces of teeth. There are many different types of whitening strips. Specific whitening strip products have their own set of instructions however the strips are typically applied twice daily for 30 minutes for 14 days. In several days, tooth colour can lighten by 1 or 2 shades. The tooth whitening endpoint does depend on the frequency of use and ingredients of the product.
   \clearpage
   
   \large
    \section{Pulse Oximetre:}
    \begin{figure}[h]
    	
    	\includegraphics[scale=0.5]{pulse.png}
    \end{figure}

Pulse oximeter is a test used to measure the oxygen level of the blood. It is
an easy, painless measure of how well oxygen is being set to part of your body
furthest from your heart, such as arms and legs.
A clip like devices called a probe is placed on a body part such as finger
or ear lobe. The probe use to measure the depth how much oxygen is in
blood. This information helps the health care to provide decide if a person
needs extra oxygen.

Pulse oximeter based on principle of Photoplethysmography. It uses light absorption phenomenon in biological tissue. When finger strip is
illuminated by red and infrared light depending on the oxygenated blood
content either red or infrared light is detected by photodetector.
Oxygenated haemoglobin absorbs more infrared light while deoxygenated
haemoglobin absorb more red light  because of their different absorption coefficient spectra at 660nm red light absorption coefficient and at 940nm.Infrared light coefficient can be obtained. Ultimately the amount of oxygen bounded with haemoglobin can be decided.
\clearpage


Medical professionals may use pulse oximeter to monitor the health of people with condition that affect blood oxygen levels, especially while they’re
in the hospital. These can include:

1-Asthma

2-Pneumonia 

3-Lung cancer


4-Anemia

5-Heart failure 

6-Congential heart disease
\\


Doctor use pulse oximetry for a number of different reasons, including:

1-To assess how well a new lung medication is working.

2-To evaluate whether someone needs help breathing.

3-To evaluate how helpful a ventilator is.

4-To monitor oxygen levels during or after surgical procedure that requires
sedation.

5-To determine whether someone needs supplemental oxygen therapy.

6-To determine how effective supplemental oxygen therapy is specially
when treatment is new.
\\

Pulse oximeter formed to be one of the important respiratory monitoring methods. In this method it is possible to determine the percentage of oxygen saturated with hemoglobin but when other possible factor like carboxyhaemoglobin and methemeoglobin are present, then to detect those fac tors need to use four wavelength to determine the fractional Sp02. Oxymeter has number of limitations which may lead to inaccurate reading  dyes, nail polish, ambient light, false alarm, skin pigmentation, low perfusion state are the different
factors which effect the reading. 
\clearpage

\large
\section{Exoskeleton:} 
\begin{figure}[h]
	\includegraphics[scale=.3]{exoskeleton.png}
\end{figure} 

Robotic exoskeletons or powered exoskeletons are considered wearable robotic units controlled by computer boards to power a system of motors, pneumatics, levers, or hydraulics to restore locomotion. The topic of exoskeletons is timely given the number of devices currently being studied as well as purchased by facilities for rehabilitation purposes in medical centers or for home use. Exoskeletons have emerged as an advantageous rehabilitation tool for disabled individuals with spinal cord injury . Rehabilitation specialists, clinicians, researchers, and patients welcome their use for over ground ambulation. Compared to previously existing locomotor training paradigms, exoskeletons may offer a great deal of independence in medical centers and communities including shopping malls, local parks and movie theaters as well as improving the level of physical activity. There is a pressing need for this population to improve their levels of physical activity. This feature may encourage continuous usage of exoskeletons in conjunction with wheelchairs.

Different powered exoskeletons are now commercially available for SCI(spinal chord injury) rehabilitation with different levels of injury. However, there is still a limited accessibility to exoskeletons in clinical settings, partly because of their prohibitive cost and the high level of training required before supervising individuals with SCI. Despite these limitations, limited research and anecdotal evidence support the use of exoskeleton to improve quality of life and health related medical conditions after SCI.

 It is crucial before expanding the applications of exoskeletons that we carefully analyse the available research and clinical evidence regarding this technology. Considering the limited data and/or small sample size of the current published studies, it is premature to draw solid conclusions about the efficacy of exoskeletons in maximizing rehabilitation outcomes or ameliorating several of the health-related consequences following SCI.

However, clinical trials are underway to confirm these benefits and to understand the underlying mechanisms that lead to such improvement. Clinical trials site (clinctrials.gov) indicated that out of 870 studies for SCIs, there are 28 studies addressing different applications of exoskeletons in this population. These statistics may highlight our limited knowledge and the need for additional clinical trials to address the major limitations of exoskeletons. The current use of robotic exoskeletons remains investigational and premature to decide whether exoskeletons are clinically effective in the rehabilitation of persons with SCI.


Persons with tetraplegia represent 55 percent of the SCI population. The level of injury cut-off was set because reasonable hand functions are required to hold the assistive device (walker or crutches) and to initiate weight shifting during stepping and walking.

With the increase in research the limitations will be removed and it will be a great contribution in medical field.
\clearpage


\section{Tissue Homogenizer:}
\begin{figure}[h]
	\includegraphics[scale=1]{tissue.png}
\end{figure}
\medspace

Homogenization to break tissues down into their constituent pieces is a common first step in the lab. Depending on the desired constituent parts, different techniques may be used. For more thorough homogenization, blending of the tissue is often done first and a disruptor is then used to break the tissue down even further. Homogenizers can have dedicated functions, such as a tissue-specific homogenizer, or more general functions. If only one type of homogenization is going to be done in the lab, a tissue homogenizer may be the most economical option, otherwise investing in a homogenizer with multiple applications may be something to consider.

Tissue homogenization is performed regularly in labs across the world for 
cell and tissue preparation. This process involves lysing the cells to 
release intracellular contents of interest, such as proteins and nuclear 
components.
\\
There are often many different names for the same piece of mechanical 
homogenizing equipment, including Cell Lysor, Disperser, High Shear Mixer, 
Homogenizer, Polytron, Rotor Stator Homogenizer, Sonicator or Tissue 
Tearor.
\\

Cell fractionation is done by homogenizer to release the organelles from 
cell. Whereas older technologies just focused on the disruption of the 
material, newer technologies also address quality or environmental aspects, 
such as cross-contamination, aerosols, risk of infection, or noise. 
Homogenization is a very common sample preparation step prior to the 
analysis of nucleic acids, proteins, cells, metabolism, pathogens, and many 
other targets.
There are several methods of cell homogenization that are commonly used. 
Continue reading for a brief explanation of each type of cell 
homogenization:
Mechanical Disruption-
Involve the use of rotating blades to grind and disperse cells , most 
effective at homogenizing cell tissues, such as liver, can homogenize small 
volumes, up to 20 liters ,sample loss is minimal, sonication.
Physical disruption used to lyse cells uses high frequency sound waves to 
lyse cells, bacteria, and other tissue types, time consuming and best used 
for small volumes (<100mL),manual grinding.

Mortar and pestle is used to manually grind cells, best suited for 
breaking apart plant tissue cells, best used for small volumes, time 
consuming, liquid homogenization.
Most widely used cell disruption technique, cells are lysed by the action 
of being forced through a small space, suitable for use with small volumes 
as well as cultured cells, liquid homogenization is the most commonly used 
homogenization technique. In the world of liquid homogenization, there are 
several types of homogenizers that are designed to complete the task: 
Potter-Elvehjem homogenizers, French Presses, and the Dounce Homogenizer.
\\

In addition, we have extensive experience in the challenges that our 
customers face as they transition from concept, through to R and D, 
clinical trials, all important FDA approval and finally, to manufacturing. 
\clearpage

\section{Camera Pill:}
\begin{figure}[h]
	\includegraphics[scale=.2]{pill.png}
\end{figure}

The advancement of technology today has enabled us to effectively use many devices in medical field. One effective and purposeful application of the advancement of technology is the process of Endoscopy, which is used to diagnose and examine the conditions of the gastrointestinal tract of the
patients. 


It is slightly larger than normal capsule. The patient swallows the capsule and the natural muscular waves of
the digestive tract propel it forward through stomach, into small intestine, through the large intestine, and then
out in the stool. It takes snaps as it glides through digestive tract twice a second. The capsule transmits the
images to a data recorder, which is worn on a belt around the patient's waist while going about his or her day
as usual. The physician then transfers the stored data to a computer for processing and analysis. The complete
traversal takes around eight hours and after it has completed taking pictures it comes out of body as excreta.
Study results showed that the camera pill was safe, without any side effects, and was able to detect
abnormalities in the small intestine, including parts that cannot be reached by the endoscope The tiniest
endoscope yet takes 30 two-megapixel images per second and offloads them wirelessly.

The primary advantages of pill camera is that its a non-invasive and comprehensive diagnostic tool. The
disposable capsule, no larger than a vitamin pill, has a tiny cameracomplete with its own lens and light source
mounted inside. As the pill travels all the way through your digestive system, it takes pictures which are sent
as images to a data recorder you wear on a waist belt. These images tell us more about your digestive system -
from the inside - than any other technology available.
With capsule endoscopy, there is no pain or discomfort. There's no sedation, surgery or hospital stay required
and the preparation is minimal. We obtain a comprehensive "picture" of your digestive system, focusing in on
the small bowel area. After your test is complete, you will return and we will download the images from the
data recorder to a computer and view a colour video of the pictures taken by the capsule. It is painless and have
no side effects. It avoids the risk in sedation. It is very efficient than X-ray CT-scan, normal endoscopy.
\\

Other advantages include:

1)No scar – as a natural body opening is used.

2)Quick recovery time.

3)Less time in hospital.

4)Often, no time in hospital is required as the procedure is performed in the doctor’s rooms.
\\

Wireless capsule endoscopy represents a significant technical breakthrough for the investigation of small
bowel, especially in light of the shortcomings of other available techniques to image this region. Though
nanotechnology has not evolved to its full capacity yet the first rung of products has already made an impact. Scientists see a lot in nanotechnology like,
Nanotechnology can be used to make miniature explosives, which would create havoc in human lives





\end{document}
\documentclass[a4paper,12pt]{article}
\usepackage{graphicx}
\graphicspath{{images/}}
\usepackage[x11names]{xcolor}
%\pagecolor{Cornsilk3}
%inline
{\Huge \title{Basic Biomedical Assignment-VI \\5 Solutions to Covid19 provided by Biomedical Engineers​}}
\author{Aashutosh Patel \\ Roll no- 21111001}
\usepackage[landscape]{geometry}
\begin{document}
	\begin{figure}
		\centering
		\includegraphics[scale=0.6]{nitlogo.png}
	\end{figure}
\maketitle {\huge \centering \title {5 Solutions to Covid19 by Biomedical Engineers:}}


	\section{Artificial Intelligence In Covid Management:}
	Artificial intelligence-based solutions assist healthcare organisations in
	coping with and combating viruses. It could be used to predict
	forthcoming pandemics or epidemics at an early stage, before they spread.
	It is feasible to anticipate and track patients by studying data. It could
	also be used to create and test novel vaccinations, as well as gain a better
	knowledge of result.
	\\
	Few notable uses are:
	\\
 1.AI is used for drug delivery design and development for vaccines.
\\
2.Monitoring the treatment and the global cases distribution.
\\
3.Early detection and diagnosis of the infection.

\section{Patient Monitoring:}
An essential element of the ICU equipment is the monitoring equipment that keeps
track of some of the patient vitals especially when they are ventilated and sedated
but also during their recovery phase to ensure the regime of ventilation is optimised
for their condition. Ventilators already provide their set of patient parameters, but
usually patient monitors are separate devices as they continue to be useful after the
patient can resume breathing on their own unassisted.
\\
 One of the key parameters for
COVID-19 patient is the amount of oxygen in their bloodstream (SpO2), measured
by pulse oximetry which uses optics within a finger clamp. Pulse oximetry tends
to be used for the duration of the patient’s stay in ICU. Modern patient monitors
provide many more patient parameters all the way to breathing waveforms to enable
clinicians to fine tune their care of the patients.
\\
Because of the risk of infection and quick transmission, the development
of software and technical applications, such as telemedicine to watch the
virus’s evolution in the population, has gotten a lot of interest.

\section{Mental Health:}
People are expected to quarantine and self-isolate, closing themselves off
socially, as the number of mental health concerns rises amid the Covid-19
health crisis. For overcoming this many apps were developed like covid coach
. 
\\
Even many people were using AR and VR for their entertainment purpose.
Doing exercise and yoga to keep themselves fit and while doing this the smart
watch records all their biological activities for tracking their health
\\
\section{Solution to Health Disparities:}
A
 systems approach
  to health disparities 
  by engineers affords
a special opportunity
    to merge the development of innovative technologies with unmet
health needs fueled by structural racism and social determinants of health.
Without
 a lens
  through
   which
    inequities
     that shape
      disease
       processes
        and
access to care,
         designs that are accessible and 
         fully
        effective for all populations
will be lacking. 
\\
Collaborations between engineers and clinicians through
clinically
 centered experiences—also known as clinical immersion
  and biodesign allows
   for the collaborative 
  identification of unmet needs and pursuit
of technical solutions to meet those needs. This can be done in a number
of training settings ranging from courses co-led by engineering and medical
schools to intensive short-term programs. The author has led a program
called
 Coulter College
  funded by the Wallace H.
   Coulter Foundation and 
   recently hosted by Medtronic. During an intensive
    3-day
  program focused on
the translation of biomedical innovations, student teams are guided through
a 
dynamic process to develop solutions to clinical needs while gaining a better understanding of resource constraints and disparities that must be considered during the design
 process.
\\
 Students learn how to evaluate the best point
of leverage within a given clinical need, how to evaluate solutions, and how
to balance clinical benefits alongside a viable commercial model. Efforts like
Coulter College and training that combines clinical immersion and biodesign
will benefit from expansion and a constant focus on underserved disparity
populations.

\section{PPE Kit:}
According to infectious disease experts, face shields protect the face from
fluids, spray, and droplets, while extending the life of N95 face masks.
The COVID-19 pandemic has depleted supplies of personal protective equipment (PPE) for healthcare professionals nationwide. Dr. Karilyn Larkin is a
hematologist at The Ohio State University Comprehensive Cancer Center –
Arthur G. James Cancer Hospital and Richard J. Solove Research Institute.
When she and her colleagues experienced shortages of face shields, she turned
to Ohio State engineers—specifically Mechanical and Aerospace Engineering
Professor Carlos Castro—for help 3D printing face shields.
	
\end{document}
\documentclass[a4paper,12pt]{article}
\usepackage{graphicx}
\graphicspath{{images/}}
\usepackage[x11names]{xcolor}
%\pagecolor{Cornsilk3}
%inline
{\Huge \title{Basic Biomedical Assignment-III \\Future of Healthcare}}
\author{Aashutosh Patel \\ Roll no- 21111001}
\usepackage[landscape]{geometry}
\begin{document}
	\begin{figure}
		\centering
		\includegraphics[scale=0.6]{nitlogo.png}
	\end{figure}
	\maketitle
	\clearpage
	
	{\huge \centering \title{Future of Healthcare:}}
	\\
	{\title{“The future of healthcare will challenge and will lead to improve the current status quo of medical accessibility, effeciency and distribution around the world”}}

\begin{figure}[h]       
	\includegraphics[scale=.4]{pic.jpeg}
\end{figure} 
\section{Introduction:}

Continuous research in various domains introduces us with lot of new technologies and also the improvement in traditional technologies. Healthcare today is not the same as it was a decade ago and it will not be same after a decade, introduction of many new technologies and effecctive application of present technologies will change this domain a lot.
\\
We do not know what new technologies we are going to discover but we can surely hope for the innovative application of some present technologies in the domain of healthcare.
\section{Modern Technologies Improving Healthcare:}

\subsection{Telehealth:}
Telehealth has made it possible for patients to receive care without an in-person office visit. In addition, remote patient monitoring is becoming more widely accepted. Having exponentially grown in popularity throughout the pandemic, this now includes wearable technology with impressive capabilities, from remote monitoring of vitals to remote echocardiograms. It wasn't accepted at this extent in healthcare industry a decade ago but today both the doctor and patient have faith in telehealth.
\\
Although the belief in remote healthcare increased in pandemic but we could trust on it because we have improved a lot in communication and other technologies used here like:
\\
i. high speed data transmission
\\
ii. improvement in data analytics
\\
Yet we have done a huge improvement in this field but still we aspire to improve and introduce lot things and technologies like 5G and Quantum computing will definitely revolutionise this field.
\\

\subsection{Artificial Intelligence:}
From making more accurate diagnoses, finding links between genetic codes to powering surgical robots, maximising administrative efficiency, and understanding how patients will respond to treatment plans, there are limitless opportunities to using Al in healthcare.
\\
With the expectation of huge potential application the use of Artificial Intelligence has already started in various ways like:
\\
i. Improving Diagnostics: It is one of Al's most exciting healthcare applications. Al solutions are helping automate image analysis and diagnosis, removing the possibility of human error in readings.
\\
ii. Drug Discovery. Al is being harnessed to identify new theraples from vast databases of information on existing medicines. This could help improve lengthy timelines and processes tied to discovering and taking drugs.
\\
iii. Primary Care: Direct to patient solutions via voice or chat-based interaction for basic medical issues. 
\clearpage
Along with these current application a lot of work is going on to improve and make more use of this technology like:
\\
Al Robot-Assisted Surgery: It is another area that is being explored to help with everything from minimally invasive procedures to open heart surgery. Working with doctors, robots have already been able to carry out complex procedures successfully with precisions flexibility, and control that goes beyond human capabilities.
\\

\subsection{Vocal Biomarkers:}
Diseases can affect organs such as the heart, lungs, brain, muscles, or vocal folds, which can then alter an individual’s voice. Therefore, voice analysis using artificial intelligence opens new opportunities for healthcare.
\begin{figure}[h]       
	\includegraphics[scale=1]{vocal.jpg}
\end{figure} 
\\
 From using vocal biomarkers we can perform many tasks like diagnosis, risk prediction and remote monitoring of various clinical outcomes and symptoms.
 \\
 a vocal biomarker is a signature, a feature, or a combination of features from the audio signal of the voice that is associated with a clinical outcome and can be used to monitor patients, diagnose a condition, or grade the severity or the stages of a disease or for drug development.
 \\
 There are many areas where the research for the application of vocal biomarker is going on and we can expect some very fascinating voice analytical  technologies in coming future.
 \\
 Some of the near future application of this technology could be in:
 \\
 i. Parkinson’s Disease
 \\
 ii. Alzheimer’s Disease and Mild Cognitive Impairment
 \\
 iii. Multiple Sclerosis and Rheumatoid Arthritis
 \\
 iv. Mental Health and Monitoring Emotions
 \\
 v. Cardiometabolic and Cardiovascular Diseases
\\
\newline
\\
\\
\\
\\
.........as the need changes as and the technology will evolve in future, the healthcare will see a lot of changes.  
\\

{\huge \title{“It is not the strongest of the species that survives, nor the most intelligent, but the one most responsive to change.”
 \begin{flushright}-Charles Darwin \end{flushright}   }}


\end{document}
